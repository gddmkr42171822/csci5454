\documentclass[12pt]{article}

\title{CSCI 5454 Final Project: AVL Tree}
\author{Robert Werthman}
\date{}

\begin{document}

\maketitle

\newpage
\tableofcontents

\newpage
\addcontentsline{toc}{section}{Introduction}
\section*{Introduction}

\addcontentsline{toc}{subsection}{What is an AVL Tree?}
\subsection*{What is an AVL Tree?}

An AVL tree is a binary search tree that is ``self-balancing''.  This means
after each operation on the tree like insertion or deletion, the heights of each
node's children differ by at most 1.  The height of a node is the number of
nodes in the longest path from a root node to it.  A root node would have a
height of 0 while its parent, if the root node was its only child, would have a 
height of 1.

\addcontentsline{toc}{subsection}{What problems does it solve?}
\subsection*{What problems does it solve?}
AVL trees, like binary trees, are used for sorting, storing, and retrieving
information.  Their advantage is that they can perform these operations faster
than if the information was stored in an array.

\addcontentsline{toc}{section}{Mathemetical Analysis of Correctness, Runtime, and Space}
\section*{Mathemetical Analysis of Correctness, Runtime, and Space}

\addcontentsline{toc}{subsection}{Correctness}
\subsection*{Correctness}

\addcontentsline{toc}{subsection}{Runtime}
\subsection*{Runtime}

\addcontentsline{toc}{subsection}{Space}
\subsection*{Space}

\addcontentsline{toc}{section}{Numerical Characterization of Runtime and Space}
\section*{Numerical Characterization of Runtime and Space}

\addcontentsline{toc}{subsection}{Description of the code invloved in the Numerical Characterization}
\subsection*{Description of the code invloved in the Numerical Characterization}

\addcontentsline{toc}{subsection}{Characterization of Runtime}
\subsection*{Characterization of Runtime}

\addcontentsline{toc}{subsection}{Characterization of Space}
\subsection*{Characterization of Space}

\addcontentsline{toc}{section}{Extensions, Improvements, and Recent Work}
\section*{Extensions, Improvements, and Recent Work}

\newpage
\addcontentsline{toc}{section}{References}

\bibliographystyle{unsrt}
\bibliography{sample}


\end{document}