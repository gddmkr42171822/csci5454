\documentclass[12pt]{article}
\usepackage{amsmath}
\usepackage{tikz}
\usepackage[linesnumbered,lined,boxed,commentsnumbered]{algorithm2e}
\begin{document}
\title{CSCI 5454: PS1}
\author{Robert Werthman}
\date{}
\maketitle

\section*{1.}

\subsection*{}
Let's say these algorithms solve an array sorting problem.\\
\begin{itemize}
\item Let algorithm $A$ be bubblesort with a worst-case runtime of $n^2$.\\
\item Let algorithm $B$ be mergesort with a worst-case runtime of $n*log(n)$.\\
\item Let $C$ be the newly designed sorting algorithm with a worst-case runtime of $h(n)$.\\
\end{itemize}
In this case, $O(min(f(n),g(n)))$ will become $O(n*log(n))$ because it is the smaller of the two runtimes.\\
If $h(n)$ is $log(n)$ then $h(n)$ achieves the running time $O(min(f(n),g(n)))$ because $log(n)$ does not grow faster than $n*log(n)$ and is therefore bounded above by it.\\

\subsection*{}
Yes, you can achieve a running time exactly $min(f(n),g(n))$. Algorithm $C$ would need to be designed in such a way that its running was equal to $min(f(n),g(n))$.\\

\section*{2.}

\subsection*{}
\textbf{Proposition/Claim: } For any real constants $a$ and $b$, where $b > 0$, the asymptotic relation $(n+a)^b = \Theta(n^b)$ is true.\\
\\
\textbf{Theorem: }The asymptotic relation $(n+a)^b = \Theta(n^b)$ is true iff:
\begin{itemize}
\item There exists positive constants $c_1, c_2, n_0$ s
uch that $0 \le c_1(n^b) \le (n+a)^b \le c_2(n^b)$ for all $n \ge n_0$.
\end{itemize}In order to prove the proposition above we must find some constants $c_1, c_2, n_0$ to satisfy the above bulleted sentence.\\
\\
\textbf{Proof: }\\
First we want to find the floor and ceiling of $n+a$ so we can create an inequality similar to the one in the theorem above.
\begin{enumerate}
\item If $|a| \le n$ then we can say that $n+a \le n+|a| \le 2n$ (Ceiling of $n+a$).
\item If $|a| \le \frac{1}{2}n$ then we can say that $n+a \ge n-|a| \ge \frac{1}{2}n$ (Floor of $n+a$). 
\end{enumerate}
Now if $2|a| \le n$ then we can combine the floor and ceilings into an compound inequality that holds true :
$$
0 \le \frac{1}{2}n \le n+a \le 2n
$$
The only thing missing from this new equation is a power of $b$.  Raising the new equation to a power of $b$ gives:
$$
0 \le (\frac{1}{2}n)^b \le (n+a)^b \le (2n)^b \Rightarrow 0 \le (\frac{1}{2})^bn^b \le (n+a)^b \le (2)^bn^b
$$  
Extracting the constants $c_1,c_2,n_0$ from this equation yields $c_1 = (\frac{1}{2})^b$, $c_2 = 2^b$, and $n_0 = 2|a|$ since $n \ge 2|a|$.  These represent one solution.
\section*{3.}
$f(n) = \Omega{g(n)}$ means that for all values to the right of some $n_0$ the value of $f(n)$ is on or above $cg(n)$.\\
\begin{center}
\begin{tabular}{|c|c|c|c|c|c|c|c|c|c|c|c|}
\hline
$n!$&$e^n$&$(\frac{3}{2})^n$&$(lg\,n)!$&$n^2$&$n\,lg\,n$&$lg(n!)$&n&$(\sqrt{2})^{lg\,n}$&$2^{lg*n}$&$n^{1/lg\,n}$&1\\
\hline
\end{tabular}
\end{center}
\subsection*{Equivalence Classes}
$lg(n!) = \Theta(n\,lg\,n)$\\
$n^{1/lg\,n} = \Theta(1)$

\section*{4.}
\subsection*{a.} $T(n) = T(n-1)+n,\,T(1) = 1$\\
I will use a recurrence tree to solve this recurrence relation.\\
\begin{center}
\begin{tikzpicture}
\node (z) {$n$}
child {node (a) {$n-1$} 
child {node (b) {$n-2$}
child {node (c) {$2$}
child {node (d) {$\vdots$}
child {node (e) {$1$}}
}
}
}
};
\end{tikzpicture}
\end{center}
The height of the tree is $n$ and the cost at the root starts at $n$ and decreases by 1 each level in the tree.\\
This means that the total cost of the tree is $n$.\\
So $T(n) = O(cost*depth) = O(n^2)$.
\subsection*{b.} $T(n) = 2T(n/2)+n^3,\,T(1) = 1$\\
I will use the master method to solve this recurrence relation.\\
$a=2, b=2, f(n)=n^3$\\
so $n^{\log_{b} a} = n^{\log_{2} 2} = n$\\
This tells us that the first 2 rules of the master theorem do not apply.
\begin{enumerate}
\item $f(n) \ne O(n^{1-\epsilon})$
\item $f(n) \ne \Theta{(n)}$
\end{enumerate}
This leaves the 3rd rule of the master theorem as the solution.\\
\begin{enumerate}
\setcounter{enumi}{2}
\item $f(n) = n^3 = \Omega{(n^{1+\epsilon})}$ if $\epsilon = 1$.  And $2f(n/2) \le cf(n) \Rightarrow 2(n/2)^3 \le cn^3$ if $c=\frac{1}{2}$ and $n \ge 1$.
\end{enumerate}
Therefore, $T(n) = \Theta{(n^3)}$.

\section*{5.}
\subsection*{a.}
\begin{algorithm}
\KwData{Nearly sorted array of size n integers}
\KwResult{Completely sorted array}
\BlankLine
\For {j = 2 to A.length}{
	key = A[j]\;
	i = j - 1\;
	\While {i $>$ 0 and A[i] $>$ key}{
		A[i+1] = A[i]\;
		i = i - 1\;
	}
	A[i+1] = key\;
}
\caption{Insertion-Sort(A)}
\end{algorithm}
\textbf{Analysis: }In order to figure out the running time of Insertion Sort we need to add up the cost of each statement in the algorithm.\\
\begin{itemize}
\item If the array is of size n then the statement \textbf{for j = 2 to A.length} will execute $n$ times with a cost of $c_1$.\\
\item The statements \textbf{key = A[j]} (inserting into an array) and \textbf{i=j-1} (setting a variable) will execute $n-1$ times each with a cost of $c_2$ and $c_3$ respectively.\\
\item Since $k$ elements are unsorted in this array than any unsorted element is no more than $k$ places away from its sorted position.  This means that the statement \textbf{while i $>$ 0 and A[i] $>$ key} could be executed in the worst case $\sum_{j=2}^{n} k$ times with a cost of $c_4$.\\
\item The statements \textbf{A[i+1] = A[i]} (inserting into an array) and \textbf{i = i + 1} (setting a variable) are executed $\sum_{j=2}^{n} k - 1$ times with a cost of $c_5$ and $c_6$ respectively.\\
\item Finally, the statement \textbf{A[i+1] = key} (inserting into an array) is executed $n-1$ times with a cost of $c_7$.\\
\end{itemize}
Therefore, the equation for the runtime, $T(n)$, of insertion-sort is:\\
\begin{align*}
T(n)  & = c_1n + c_2(n-1) + c_3(n-1) + c_4(\sum_{j=2}^{n} k) + c_5(\sum_{j=2}^{n} k - 1) + c_6(\sum_{j=2}^{n} k - 1) + c_7(n-1)\\
	& = c_1n + c_2(n-1) + c_3(n-1) + c_4(k(n-1)) + c_5(\sum_{j=2}^{n} k - 1) + c_6(\sum_{j=2}^{n} k - 1) + c_7(n-1)
\end{align*}
Since $k < n$ further reduction of $T(n)$ would yield a linear function of $n$ so we can say the runtime would turn out to be $O(n)$.

\subsection*{b.}
The sorting algorithm I suggest to get a $O(n)$ runtime is Counting Sort.
\begin{algorithm}
\KwData{A is the input array of length $n$}
\KwData{B is the sorted array of length $n$}
\KwData{$k$ is the highest integer in A}
\BlankLine
let $C[0..k]$ be a new array\\
\For{$i=0$ to $k$}{
	$C[i] = 0$\\
}
\For{$j=1$ to $A.length$}{
	$C[A[j]] = C[A[j]+1$\\
}
\For{$i=1$ to $k$}{
	$C[A[j]] = C[i] + C[i-1]$\\
}
\For{$j = A.length$ downto 1}{
	$B[C[A[j]]] = A[j]$\\
	$C[A[j]] = C[A[j]] - 1$\\
}
\caption{Counting-Sort(A, B, $k$)}
\end{algorithm}
\\
\textbf{Analysis: }
\begin{itemize}
\item Initializing $C[0..k]$ takes $k+1$ time to execute and costs $c_0$.
\item The statement \textbf{for i = 0 to k} take $k+1$ times to execute and cost $c_1$.
\item The statements \textbf{for j = 1 to A.length} and \textbf{j = A.length downto 1} take $n$ times to execute and cost $c_3$ and $c_4$ respectively.
\item The statement \textbf{i = 1 to k} takes $k$ times to execute and costs $c_2$.
\end{itemize}
The equation for the runtime, $T(n)$, of Counting Sort is:\\
$$
T(n) = c_0(k+1) + c_1(k+1) + c_3n + c_4n + c_2k\ldots
$$

Reducing $T(n)$ further would show that the runtime of Counting Sort is a linear function of $n$ that runs in a linear time of $O(k+n)$.
If $k=O(n)$ then the running time is $\Theta(n)$.

\subsection*{c.}
(b) doesn't contradict the $\Omega{(n\,log\,n)}$ lower bound given on page 59 of the textbook because the algorithm is not a comparison sorting algorithm.\\
It has been proven that any comparison sort must make $\Omega{n\,log\,n)}$ comparisons in the worst case to sort $n$ elements.  Since counting sort is not a comparison sorting algorithm its runtime is not bounded by $\Omega{(n\,log\,n)}$.


\end{document}  