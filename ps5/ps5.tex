\documentclass[12pt]{article}
\setlength{\oddsidemargin}{0in}
\setlength{\evensidemargin}{0in}
\setlength{\textwidth}{6.5in}
\setlength{\parindent}{0in}
\setlength{\parskip}{\baselineskip}

\usepackage{amsmath,amsfonts,amssymb,graphicx,enumerate,float}

\begin{document}

CSCI 5454 \hfill Problem Set 5\\
Robert Werthman\\
No Collaborators

\hrulefill

\begin{enumerate}
  
  \item \textit{Formulate Gru's problem as an integer programming problem and
  write its linear programming relaxation.}\\
  \\
  \textbf{Integer Programming Problem}\\
  \\
  Assume each drone can communicate with any number of the $k$ stations. The
  constraint for this problem is that we have to use every drone but we don't
  have to use every station.  For this problem we want to minimize the sum of
  the cost of activating a station $a_j$ and the cost $c_{ij}$ of each of the
  drones communicating with that station.  In mathematical terms we are trying to 
  $$ 
  minimize\,\sum_{j=1}^{k} a_jx_j + \sum_{j = 1}^{k} x_j\sum_{i=1}^{n} c_{ij}
  $$ 
  subject to the constraints
  \begin{enumerate} 
  \item
  $$
  x_j = 
  \begin{cases}
  1 & \text{ if we activate station }j\\
  0 & \text{ if we don't activate station }j\\
  \end{cases}
  $$
  \item
  Let $d_{ij} = 1$ if drone $i$ is communicating with station $j$ 
  then the mathematical notation for ensuring all drones are communicating with
  at least one station is 
  $$
  \forall d \quad \sum_{j=1}^{k} d_{ij}x_j \ge 1
  $$
  \end{enumerate}

  \textbf{Linear Programming Relaxation}\\
  \\
  The linear programming relaxation of this problem is still trying to
  $$ 
  minimize\,\sum_{j=1}^{k} a_jx_j + \sum_{j = 1}^{k} x_j\sum_{i=1}^{n} c_{ij}
  $$
  but instead of $x_j \in \mathbb{Z}$ and being only 1 or 0 it can now be any
  number between 1 and 0.  $x_j$ is now $\in \mathbb{R}$.  The question now
  though is how do you round $x_j$ to make sure the constraint
  $$
  \forall d \quad \sum_{j=1}^{k} d_{ij}x_j \ge 1
  $$
  is still satisfied.  As was discussed in lecture you should use randomized
  rounding.
  
  \newpage
  \item \textit{Find an $s-t$ flow of amount at least $d$ that minimizes the
  total cost, or report that such a flow does not exist.}
  \begin{enumerate}
    \item \textit{Show how to solve this problem using linear programming.}\\
    \\
    In this problem we want to send the target amount of flow $d$ from $s$ to
    $t$.  At the same time we want to minimize the total cost of the units
    of flow $f_e$ for all $e \in E$.  Written mathematically this is 
    $$
    minimize \, \sum_{(u,v) \in E} p_{u,v}f_{u,v}
    $$
    subject to the contraints
    \begin{enumerate}
      
      \item 
      $$
      f_{u,v} \le c_{u,v}
      $$
      The flow of an edge is less than or equal to the capacity of the edge. 
      
      \item
      $$
      \sum_{v \in V} f_{s,v} = \sum_{v \in V} f_{v,t} = d
      $$
      The sum of the flow from the source vertex $s$ to other vertices must
      equal the target flow $d$ and the sum of the flow from other vertices to
      the sink vertex $t$ must equal the target flow $d$.
      
      \item
      $$
      \sum_{(w,u) \in E} f_{w,u} = \sum_{(u,z) \in E} f_{u,z}
      $$
      The amount of flow entering $u$ equals the amount of flow leaving $u$.
      Flow is conserved.
    \end{enumerate}
    If these constraints are satisfied by a graph with a particular flow then an
    $s-t$ flow of amount at least $d$ that minimizes the total cost exists,
    otherwise no such flow exists.
    
    
    \item \textit{Show that the $s-t$ shortest path problem and the classic
    network flow problem are both special cases of this problem.}
  \end{enumerate}

\end{enumerate}

\end{document}