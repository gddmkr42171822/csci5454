\documentclass[12pt]{article}
\setlength{\oddsidemargin}{0in}
\setlength{\evensidemargin}{0in}
\setlength{\textwidth}{6.5in}
\setlength{\parindent}{0in}
\setlength{\parskip}{\baselineskip}

\usepackage{amsmath,amsfonts,amssymb,graphicx,enumerate,float}

\begin{document}

CSCI 5454 \hfill Problem Set 5\\
Robert Werthman\\
No Collaborators

\hrulefill

\begin{enumerate}
  
  \item \textit{Formulate Gru's problem as an integer programming problem and
  write its linear programming relaxation.}\\
  \\
  \textbf{Integer Programming Problem}\\
  \\
  Assume each drone can communicate with any number of the $k$ stations. The
  constraint for this problem is that we have to use every drone but we don't
  have to use every station.  For this problem we want to minimize the sum of
  the cost of activating a station $a_j$ and the cost $c_{ij}$ of each of the
  drones communicating with that station.  In mathematical terms we are trying to 
  $$ 
  \text{minimize} \,\sum_{j=1}^{k} a_jx_j + \sum_{j = 1}^{k} x_j\sum_{i=1}^{n}
  c_{ij} $$ 
  subject to the constraints
  \begin{enumerate} 
  \item
  $$
  x_j = 
  \begin{cases}
  1 & \text{ if we activate station }j\\
  0 & \text{ if we don't activate station }j\\
  \end{cases}
  $$
  \item
  Let $d_{ij} = 1$ if drone $i$ is communicating with station $j$ 
  then the mathematical notation for ensuring all drones are communicating with
  at least one station is 
  $$
  \forall d \quad \sum_{j=1}^{k} d_{ij}x_j \ge 1
  $$
  \end{enumerate}

  \textbf{Linear Programming Relaxation}\\
  \\
  The linear programming relaxation of this problem is still trying to
  $$ 
  \text{minimize} \,\sum_{j=1}^{k} a_jx_j + \sum_{j = 1}^{k} x_j\sum_{i=1}^{n}
  c_{ij} $$
  but instead of $x_j \in \mathbb{Z}$ and being only 1 or 0 it can now be any
  number between 1 and 0.  $x_j$ is now $\in \mathbb{R}$.  The question now
  though is how do you round $x_j$ to make sure the constraint
  $$
  \forall d \quad \sum_{j=1}^{k} d_{ij}x_j \ge 1
  $$
  is still satisfied.  As was discussed in lecture you should use randomized
  rounding.
  
  \newpage
  \item \textit{Find an $s-t$ flow of amount at least $d$ that minimizes the
  total cost, or report that such a flow does not exist.}
  \begin{enumerate}
    \item \textit{Show how to solve this problem using linear programming.}\\
    \\
    In this problem we want to send the target amount of flow $d$ from $s$ to
    $t$.  At the same time we want to minimize the total cost of the units
    of flow $f_e$ for all $e \in E$.  Written mathematically this is 
    $$
    \text{minimize} \, \sum_{(u,v) \in E} p_{u,v}f_{u,v}
    $$
    subject to the contraints
    \begin{enumerate}
      
      \item 
      $$
      f_{u,v} \le c_{u,v}
      $$
      The flow of an edge is less than or equal to the capacity of the edge. 
      
      \item
      $$
      \sum_{v \in V} f_{s,v} = \sum_{v \in V} f_{v,t} = d
      $$
      The sum of the flow from the source vertex $s$ to other vertices must
      equal the target flow $d$ and the sum of the flow from other vertices to
      the sink vertex $t$ must equal the target flow $d$.
      
      \item
      $$
      \sum_{(w,u) \in E} f_{w,u} = \sum_{(u,z) \in E} f_{u,z}
      $$
      The amount of flow entering $u$ equals the amount of flow leaving $u$.
      Flow is conserved \cite{2}.
    \end{enumerate}
    If these constraints are satisfied by a graph with a particular flow then an
    $s-t$ flow of amount at least $d$ that minimizes the total cost exists,
    otherwise no such flow exists.
     
    \item \textit{Show that the $s-t$ shortest path problem and the classic
    network flow problem are both special cases of this problem.}\\
    \\
    The classic network flow problem seeks to maximize the flow from $s$ to $t$.
    This is the same as the problem $2a$ but in this case we want to
    maximize $f_e$ for all $e \in E$ by letting $d$ be the maximum flow possible
    in the graph.  We also no longer suffer a penalty $p_{u,v}$ per unit flow
    for the classic network flow problem as we do in the problem from $2a$.\\
    \\
    The $s-t$ shortest path problem seeks to minimize the sum of the cost of the
    edges involved in the flow $s-t$.  In this case we no longer have a target
    flow $d$ as in the problem $2a$ but were are still trying to minimize the
    penalty/cost $p_{u,v}$ per edge per unit flow in the graph.
  \end{enumerate}
  
  \newpage
  \item \textit{Give a polynomial time 3-approximation algorithm that, given an
  undirect graph $G$ with nonnegative edge weights, removes a set of edges of
  minimum total weight from $G$ so the remaining graph is triangle free. 
  Return the maximum-weight subgraph which is triangle free.}\\
  \\
  \textbf{Algorithm}\\
  \\
  I believe you can solve this problem with linear programming specifically
  integer linear programming.
  We wish to minimize the total weight of the edges we remove from the graph.  
  This can
  written as
  $$
  \text{minimize} \sum_{(u,v) \in E} w(u,v)x_{u,v}
  $$
  subject to the constraints
  \begin{enumerate}
  \item
  $$
  x_{u,v} = 
  \begin{cases}
  1 & \text{ if we choose to remove edge }(u,v)\\
  0 & \text{ if we don't choose to remove edge }(u,v)\\
  \end{cases}
  $$ 
  
  \item
  $$
  \forall\{(u,v),(v,x),(u,x)\} \quad x_{u,v} + x_{v,x} + x_{u,x} \ge 1
  $$
  This says that for all edges that form a triangle at least one edge has to be
  removed in order for there not to be a triangle anymore.
  \end{enumerate}
  First, we need to find all of the triangles in the graph $G$ and then remove
  an edge so the traingle no longer exists.
  This can be done by comparing the adjacency list of vertex $u$ with the
  adjacency lists of the vertices in $u$'s adjacency list.  For example if $u$'s
  adjacency list was $[v,x]$, $v$'s adjacency list was $[x,u]$, and $x$'s
  adjacency list was $[u,v]$ there would be a triangle in the $G$ between
  $(u,v),\,(v,x),\,(x,u)$.  The psuedocode would look like
  \begin{verbatim}
  def FindAndRemoveTraingles(G):
    for each u in V:
      for a in u's adjacency list:
        for b in a's adjacency list:
          # first iteration: a = v, b = x
          if u in b's adjacency list:
            # There is a triangle between (u,a), (a,b), (b,u)
            # Compare weight of edges and remove lowest weight edge
            # Update adjacency lists
     return G
  \end{verbatim}
  
  \textbf{Correctness and Runtime}\\
  \\
  The worst case graph $G$ for this problem is a graph 
  that is connected
  which means all vertices have edges to all other vertices in the graph.
  This maximizes the number of triangles a graph will have.\\
  \\
  The base case is 
  when there are 3 vertices in $G$ which forms a single triangle. 
  FindAndRemoveTriangles will remove a single edge from the triangle that has
  the lowest weight.
  Once that edge is removed there is no longer a triangle and the graph is
  triangle free.\\
  \\
  Inductively we can say that since the algorithm removes
  a single triangle it will remove all of the triangles in a graph.  This is
  true because FindAndRemoveTriangles visits every vertex and finds every 
  triangle in the graph.  
  For every one of these triangles an edge is removed thus removing the
  triangle.  This will leave any graph with any number of triangles triangle 
  free.\\
  \\
  Because the graph is connected the size of each adjacency list
  would be $|V|-1$.  The runtime of FindAndRemoveTraingles will then 
  be $O(|V|^3)$.\\
  \\
  \textbf{Approximation Proof}\\
  \\
  Since we are trying to minimize for this problem in order to prove that this
  algorithm is a 3-approximation algorithm, we have to show that the value of 
  the solution $x$
  which is the total weight of all the edges removed by the algorithm
  is at most 3 times the value of the optimal solution.  This statement shown by
  the inequality 
  $$
  \frac{ALG}{OPT} \le 3
  $$
  \\
  As stated above we found a solution with integer linear programming.  But what
  if we relaxed the ILP problem to just a LP problem.  This means that $x_{u,v}$
  is now $\in \mathbb{R}$ and could be any real number between and including 0
  and 1.  What if we also said that the solution to the ILP problem was the
  optimal solution $OPT$.  This means that $LP \le ILP = OPT$ \cite{1}.\\
  \\
  The constraints we have for the $LP$ problem are
  \begin{enumerate}
    \item 
    $0 \le x_{u,v} \le 1$
    
    \item
     $$
     \forall\{(u,v),(v,x),(u,x)\} \quad x_{u,v} + x_{v,x} + x_{u,x} \ge 1
     $$
     This says that for all edges that form a triangle at least one edge has to
     be removed in order for there not to be a triangle anymore.  
  \end{enumerate}
  In the LP
  problem though we are now dealing with possible fractions for $x_{u,v}$. 
  This means we will have to round those fractions to a whole number in order
  for the LP problem solution to still remain a solution to the ILP problem.  
  To do this we define $x'_{u,v} = 1$ if $x^{*}_{u,v} \ge 1/3$ and $x'_{u,v}
  = 1$ if $x^{*}_{u,v} < 1/3$ \cite{1}.\\
  \\
  Using this rounding scheme we can show that the $LP$ is a 3-approximation
  algorithm.  The minimal weight produced by the $LP$ solution is
  \begin{align*}
  &\sum_{(u,v) \in E} w(u,v)x'_{u,v}\\
  &\le \sum_{(u,v) \in E} 3w(u,v)x^{*}_{u,v}\\
  &= 3\sum_{(u,v) \in E} w(u,v)x_{u,v}\\
  &= 3 \, OPT
  \end{align*}
  This shows that 
  $$
  \frac{LP=ALG}{ILP=OPT} \le 3
  $$
  which means the LP algorithm is a 3-approximation algorithm.
    
\end{enumerate}

\newpage
\begin{thebibliography}{2}
\bibitem{1} http://www.eecs.berkeley.edu/~luca/cs261/lecture07.pdf
\bibitem{2} http://beust.com/algorithms.pdf
\end{thebibliography}

\end{document}